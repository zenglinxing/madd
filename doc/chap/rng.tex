\chapter{random number generator (RNG)}

All the RNG algorithms have such struct and functions are in the form of follows. $<$algorithm$>$ should be replaced by the algorithm's name, as listed in the next section. But do note some algorithms may have different functions, which we will introduce in the following context.
\begin{lstlisting}[language=C, title={Definition \& functions of RNG.},]
typedef struct ... RNG_<algorithm>_Param;
RNG_<algorithm>_Param RNG_<algorithm>_Init(uint64_t seed);
/* whether U64 or U32, depends on the algorithm */
uint64_t/uint32_t RNG_<algorithm>_U64/U32(RNG_<algorithm>_Param *rng);
double Rand_<algorithm>(RNG_<algorithm>_Param *rng);
float Rand_<algorithm>_f32(RNG_<algorithm>_Param *rng);
long double Rand_<algorithm>_fl(RNG_<algorithm>_Param *rng);
RNG_<algorithm>_Param RNG_<algorithm>_Read_BE(FILE *fp);
RNG_<algorithm>_Param RNG_<algorithm>_Read_LE(FILE *fp);
void RNG_<algorithm>_Write_BE(RNG_<algorithm>_Param *rng, FILE *fp);
void RNG_<algorithm>_Write_LE(RNG_<algorithm>_Param *rng, FILE *fp);
\end{lstlisting} 

\section{$<$algorithm$>$s}

Here lists the algorithms and the corresponding $<$algorithm$>$ in the struct and functions.

Mersenne Twister: $<$algorithm$>$=MT. Here only apply the MT19937-64 parameters. 

C library: $<$algorithm$>$=Clib. The default RNG in C standard library. It has no \_Read\_ and \_Write\_ functions.

x86: $<$algorithm$>$=x86. The RNG by x86 CPU. So only when Madd is built on x86 platform could it be used. It has no \_Read\_ and \_Write\_ functions.

Xorshift64 / Xorshift64*: $<$algorithm$>$=Xorshift64/Xorshift64s. They share the same RNG\_Xorshift64\_Param.

Xorshift1024*: $<$algorithm$>$=Xorshift1024s.

Xoshiro256+ / Xoshiro256**: $<$algorithm$>$=Xoshiro256p/Xoshiro256ss.

Xorwow: $<$algorithm$>$=Xorwow. Note its function name \textbf{RNG\_Xorwow\_U32}.

\section{example}

Take MT algorithm as an example.

\begin{lstlisting}[language=C, title={Example of RNG MT.},]
#include<stdio.h>
#include<madd.h>

int main(int argc, char *argv[])
{
    uint64_t seed = 10;
    RNG_MT_Param rng = RNG_MT_Init(seed);
    
    // generate random number
    double array[8];
    uint64_t i;
    for (i=0; i<8; i++){
        array[i] = Rand_MT(&rng);
    }
    
    // save RNG
    FILE *fp = fopen("RNG_MT-BE", "wb");
    RNG_MT_Write_BE(&rng);
    fclose(fp);
    
    // load RNG
    RNG_MT_Param mt;
    fp = fopen("RNG_MT-BE", "rb");
    RNG_MT_Read_BE(&mt);
    fclose(fp);
    return 0;
}
\end{lstlisting} 