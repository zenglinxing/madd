% coding: utf-8
\documentclass[a4paper, oneside,]{book}
\usepackage[top=3cm, bottom=3cm, left=2cm, right=2cm,]{geometry}
\usepackage{times, mathptmx, listings, multirow, color, hyperref,}
\lstset{
    breaklines,
    columns=fixed,
    numbers=left,
    keywordstyle=\color{blue}\textbf,
    %identifierstyle=\color{brown!80!black},
    commentstyle=\rmfamily\itshape,
    stringstyle=\color{magenta},
    %numberstyle=\color{teal},
    showstringspaces=false,
    frame=tb,
}
\hypersetup{pdftitle={Guide of Math Addition},
            pdfauthor={Lin-Xing Zeng},
            pdfkeywords={math, C lib},
}
\title{Guide of Math Addition (Madd)}
\author{Lin-Xing Zeng}
\begin{document}
\maketitle
\tableofcontents

\chapter*{Preface}

\indent Math Addition (Madd) is a C lib for numerical computation. It could be built through CMake. It requires the C compiler to support C99 standard.

To build Madd, you should guarantee that your machine is binary 64-bit, supporting 64-bit float number.

\begin{lstlisting}[language=bash, title={Build Madd in source dir.},]
mkdir build
cd build
cmake .. -DCMAKE_INSTALL_PREFIX=<your install path> -DCMAKE_BUILD_TYPE=Release
cmake --build .
ctest
cmake --install .
\end{lstlisting}

Include Madd in C source

\begin{lstlisting}[language=C, title={include Madd header.},]
#include<madd.h>
\end{lstlisting}

And link Madd lib.

\chapter{basic}

\section{error information}

Madd supports error information logging. In details, the functions with names starting with {\it Madd\_Error} process the error/warning informations.

\begin{lstlisting}[language=C, title={Functions of error information.},]
bool Madd_Error_Enable_Logfile(const char *log_file_name);
void Madd_Error_Set_Logfile(FILE *fp);
void Madd_Error_Close_Logfile(void);
void Madd_Error_Add(char sign, const wchar_t *info);
void Madd_Error_Print_Last(void);
void Madd_Error_Save_Last(FILE *fp);
void Madd_Error_Print_All(void);
void Madd_Error_Save_All(FILE *fp);
char Madd_Error_Get_Last(Madd_Error_Item *mei); 
\end{lstlisting} 

Madd\_Error\_Add provides new information. {\it sign} only accepts MADD\_ERROR and MADD\_WARNING. {\it *info} is a wide character string, but the length is limited to MADD\_ERROR\_INFO\_LEN.

If you want to print the error/warning instantly, you can call Madd\_Error\_Print\_Last or Madd\_Error\_Print\_All to get the latest errors/warnings. But do remember Madd only remembers MADD\_ERROR\_MAX items, hence it may not print all the previous errors/warnings. 

To record more error/warning informations, you can also enable log file. Madd\_Error\_Enable\_Logfile accepts the log file name, and returns the boolean to inform whether the file is created under expectation. You can also call Madd\_Error\_Set\_Logfile to redirect the info writing to the opened file. You may not need to call Madd\_Error\_Close\_Logfile to close the log file, since Madd has already call fflush after writing the log file. However, Madd\_Error\_Close\_Logfile will at least safely save the file. 

\section{constant}

Madd includes some natural constants: Pi $\rightarrow \pi$, E\_Nat $\rightarrow e$.

There are some binary integer numbers are useful for masking, as shown in Table~\ref{table:binary mask int}.

\begin{table}[htbp]
  \centering
  \caption{Binary mask integer.}\label{table:binary mask int}
  \begin{tabular}{cc}
    \hline\hline
    type & variable \\
    \hline
    \multirow{5}*{uint8\_t} & Bin4 \\
    {} & Bin5 \\
    {} & Bin6 \\
    {} & Bin7 \\
    {} & Bin8 \\
    \hline
    \multirow{2}*{uint16\_t} & Bin15 \\
    {} & Bin16 \\
    \hline
    \multirow{2}*{uint32\_t} & Bin31 \\
    {} & Bin32 \\
    \hline
    \multirow{2}*{uint64\_t} & Bin63 \\
    {} & Bin64 \\
    \hline\hline
  \end{tabular}
\end{table}

\section{log2 integer functions}

\begin{lstlisting}[language=C, title={Functions of integer log2 functions.},]
uint64_t Log2_Floor(uint64_t x);
uint64_t Log2_Ceil(uint64_t x);
void Log2_Full(uint64_t x, uint64_t *lower, uint64_t *upper);
\end{lstlisting}

Log2\_Floor and Log2\_Ceil are similar, accepting uint64\_t and return uint64\_t. The results are just like floor(log2(a)) and ceil(log2(a)).

\section{binary union}

Madd is targeted on binary 64-bit platform. Madd defines 4 unions for different length of data types.
\begin{lstlisting}[language=C, title={Unions of data types.},]
union _union8; // 8-bit
union _union16; // 16-bit
union _union32; // 32-bit
union _union64; // 64-bit
\end{lstlisting}

union \_union8 has .u and .i in type uint8\_t int8\_t. union \_union64 has .f .u and .i in type double, uint64\_t and int64\_t. Whereas, union \_union64 also has array members: .f32[2] (float), .u8[8] (uint8\_t), .i8[8], .u16[4], .i16[4], .u32[2], .i32[2]. Things are similar in union \_union16 and union \_union32.
\chapter{random number generator (RNG)}

All the RNG algorithms have such struct and functions are in the form of follows. $<$algorithm$>$ should be replaced by the algorithm's name, as listed in the next section. But do note some algorithms may have different functions, which we will introduce in the following context.
\begin{lstlisting}[language=C, title={Definition \& functions of RNG.},]
typedef struct ... RNG_<algorithm>_Param;
RNG_<algorithm>_Param RNG_<algorithm>_Init(uint64_t seed);
/* whether U64 or U32, depends on the algorithm */
uint64_t/uint32_t RNG_<algorithm>_U64/U32(RNG_<algorithm>_Param *rng);
double Rand_<algorithm>(RNG_<algorithm>_Param *rng);
float Rand_<algorithm>_f32(RNG_<algorithm>_Param *rng);
long double Rand_<algorithm>_fl(RNG_<algorithm>_Param *rng);
RNG_<algorithm>_Param RNG_<algorithm>_Read_BE(FILE *fp);
RNG_<algorithm>_Param RNG_<algorithm>_Read_LE(FILE *fp);
void RNG_<algorithm>_Write_BE(RNG_<algorithm>_Param *rng, FILE *fp);
void RNG_<algorithm>_Write_LE(RNG_<algorithm>_Param *rng, FILE *fp);
\end{lstlisting} 

\section{$<$algorithm$>$s}

Here lists the algorithms and the corresponding $<$algorithm$>$ in the struct and functions.

Mersenne Twister: $<$algorithm$>$=MT. Here only apply the MT19937-64 parameters. 

C library: $<$algorithm$>$=Clib. The default RNG in C standard library. It has no \_Read\_ and \_Write\_ functions.

x86: $<$algorithm$>$=x86. The RNG by x86 CPU. So only when Madd is built on x86 platform could it be used. It has no \_Read\_ and \_Write\_ functions.

Xorshift64 / Xorshift64*: $<$algorithm$>$=Xorshift64/Xorshift64s. They share the same RNG\_Xorshift64\_Param.

Xorshift1024*: $<$algorithm$>$=Xorshift1024s.

Xoshiro256+ / Xoshiro256**: $<$algorithm$>$=Xoshiro256p/Xoshiro256ss.

Xorwow: $<$algorithm$>$=Xorwow. Note its function name \textbf{RNG\_Xorwow\_U32}.

\section{example}

Take MT algorithm as an example.

\begin{lstlisting}[language=C, title={Example of RNG MT.},]
#include<stdio.h>
#include<madd.h>

int main(int argc, char *argv[])
{
    uint64_t seed = 10;
    RNG_MT_Param rng = RNG_MT_Init(seed);
    
    // generate random number
    double array[8];
    uint64_t i;
    for (i=0; i<8; i++){
        array[i] = Rand_MT(&rng);
    }
    
    // save RNG
    FILE *fp = fopen("RNG_MT-BE", "wb");
    RNG_MT_Write_BE(&rng);
    fclose(fp);
    
    // load RNG
    RNG_MT_Param mt;
    fp = fopen("RNG_MT-BE", "rb");
    RNG_MT_Read_BE(&mt);
    fclose(fp);
    return 0;
}
\end{lstlisting} 
\chapter{sort}

There are multiple prevalent sort algorithm implemented in Madd. As listed in Table

\begin{table}[htbp]
  \centering
  \caption{Sort algorithms}\label{table: sort algorithm}
  \begin{tabular}{cccccc}
    \hline\hline
    \multirow{2}*{sort type} & \multirow{2}*{algorithm} & \multicolumn{2}{c}{time complexity} & \multirow{2}*{space complexity} & \multirow{2}*{stability} \\
    {} & {} & best & worst & {} & {} \\
    \hline
    integer & counting sort & $O(n+k)$ & $O(n+k)$ & $O(k)$ & Y \\
    \hline
    compare & heap sort & $O(n\log n)$ & $O(n\log n)$ & $O(1)$ & N \\
    {} & insertion sort & $O(n)$ & $O(n^{2})$ & $O(1)$ & Y \\
    {} & merge sort & $O(n\log n)$ & $O(n\log n)$ & $O(n)$ & Y \\
    {} & quick sort & $O(n\log n)$ & $O(n^{2})$ & $O(\log n)$ & Y \\
    {} & shell sort & $O(n^{7/6})$ & $O(n^4/3)$ & O(1) & N \\
    \hline\hline
  \end{tabular}
\end{table} 

Note that the shell sort function in Madd uses Sedgewick sequence, so the time complexity is smaller than what you learn from the textbook. 

The sort type indicates the type of sorted element.
If the sort type is integer, the algorithm just sort the integer.
If the sort type is compare, then you can sort any types of element.

For the algorithms sorting integer, you should prepare a get-key function.
For algorithms sorting any types, you should prepare a compare function.
They should be declared as follows.
{\it get\_key} returns an integer represents the given element.
{\it compare\_func} returns true if a is \emph{less than} or \emph{equal to b}.

\begin{lstlisting}[language=C, title={Declarations of functions for sorting.},]
uint64_t get_key(void *element, void *other_param);
bool compare_func(void *a, void *b, void *other_param);
\end{lstlisting}

\section{Declaration of Sort Functions}

\begin{lstlisting}[language=C, title={Declarations of sort functions.},]
void Sort_Counting(uint64_t n_element, size_t usize, void *arr_,
                   uint64_t get_key(void *element, void *other_param), void *other_param);

void Sort_Insertion(uint64_t n_element, size_t usize, void *arr_,
                    bool func_compare(void *a1, void *a2, void *other_param), void *other_param);
void Sort_Merge(uint64_t n_element, size_t usize, void *arr_,
                bool func_compare(void *a1, void *a2, void *other_param), void *other_param);
void Sort_Quicksort(uint64_t n_element, size_t usize, void *arr_,
                    bool func_compare(void *a, void *b, void *other_param), void *other_param);
void Sort_Shell(uint64_t n_element, size_t usize, void *arr_,
                bool func_compare(void*, void*, void*), void *other_param);
void Sort_Heap_Internal(uint64_t n, size_t usize, void *arr_,
                        bool func_compare(void*, void*, void*), void *other_param,
                        void *ptemp);
void Sort_Heap(uint64_t n, size_t usize, void *arr_,
                    bool func_compare(void*, void*, void*), void *other_param);
\end{lstlisting} 

{\it arr\_} is the pointer to your array. {\it usize} is the size of element.

The function {\it Sort\_Heap\_Internal} has one more parameter than {\it Sort\_Heap} *ptemp.
I suppose you had prepare memory space of *ptemp by {\it usize} bytes.

\section{get-key function to compare function}

You may be upset with transfering get-key function to compare function.
Madd has already implements a function to replace your get-key to compare.

\begin{lstlisting}[language=C, title={get-key function to compare function.},]
typedef struct{
    uint64_t (*get_key_func)(void*, void*);
    void *other_param;
} Sort_Key_Func_to_Compare_Func_Param;

bool Sort_Key_Func_to_Compare_Func(void *a, void *b, void *input_param);
\end{lstlisting}

\begin{lstlisting}[language=C, title={Example of get-key function to compare function.},]
#include<madd.h>
uint64_t get_key(void *a, void *other_param)
{
    uint64_t *aa = (uint64_t)a;
    return *aa;
}

int main(int argc, char *argv[])
{
    uint64_t arr[4] = {1, 2, 3, 4};
    Sort_Counting(arr, 4, sizeof(uint64_t), NULL);
    
    Sort_Key_Func_to_Compare_Func_Param *param = {.get_key_func=get_key, .other_param=NULL};
    Sort_Merge(arr, 4, sizeof(uint64_t), Sort_Key_Func_to_Compare_Func, param);
    return 0;
}
\end{lstlisting}

\section{Binary Search}

\begin{lstlisting}[language=C, title={Binary search functions.},]
uint64_t Binary_Search(uint64_t n, size_t usize, void *arr_, void *element,
                       char func_compare(void *a, void *b, void *other_param), void *other_param);
uint64_t Binary_Search_Insert(uint64_t n, size_t usize, void *arr_, void *element,
                              bool func_compare(void *a, void *b, void *other_param), void *other_param);
\end{lstlisting}

The binary-search is an efficient method to search for element in a \textbf{sorted} array.
{\it Binary\_Search} searches for the element in array, and returns where it is.
If the element is not found, it will returns the possible (maybe not accurate) place to insert the element and pops warning to Madd.
{\it Binary\_Search\_Insert} returns where to insert the element.

Important difference! {\it func\_compare} you provide to {\it Binary\_Search} should only return {\it MADD\_LESS}/{\it MADD\_SAME}/{\it MADD\_GREATER}.


\end{document} 